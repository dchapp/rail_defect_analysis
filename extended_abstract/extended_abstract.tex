
\documentclass{sig-alternate-05-2015}


\begin{document}

% Copyright
\setcopyright{acmcopyright}

% DOI
\doi{}

% ISBN
\isbn{}

%Conference
\conferenceinfo{}{}

\acmPrice{}

%
% --- Author Metadata here ---
\conferenceinfo{}{}
%\CopyrightYear{2007} % Allows default copyright year (20XX) to be over-ridden - IF NEED BE.
%\crdata{0-12345-67-8/90/01}  % Allows default copyright data (0-89791-88-6/97/05) to be over-ridden - IF NEED BE.
% --- End of Author Metadata ---

\title{Leveraging Spark and Docker for Scalable, Reproducible Analysis of Railroad Defects}

\numberofauthors{2}
\author{
% 1st. author
\alignauthor
Dylan Chapp\\
       \affaddr{University of Delaware}\\
       \email{dchapp@udel.edu}
% 2nd. author
\alignauthor
Surya Kasturi\\
       \affaddr{University of Delaware}\\
       \email{suryak@udel.edu}
}

\maketitle
\begin{abstract}
\end{abstract}

\keywords{}

\section{Introduction}
We present a scalable analysis platform for railway defect data and a proof of concept 
using two railroad defect data sets. 

Railway defects present a major threat to railway network 
integrity, which in turn has serious economic consequences.~\cite{Schafer:08}
In light of the significance of railway defects, there is consensus in the railroad engineering 
community that big data analytics platforms can contribute to prediction of 
defects.~\cite{Zarembski:14} 


\section{Methodology}

\subsection{Data}
The data that we test our platform against is partitioned into two distinct sets, one containing
only records of rail defects and the other containing only records of track geometry defects. 
For our purposes, rail defects refer to physical degradation of the rails themselves, e.g., cracks,
whereas track geometry defects refer to instances where track components have moved out of their 
correct positions.~\cite{RailroadSafety:15}

\section{Evaluation}

\section{Continuing Work}
We identified the potential of acheiving better classification accuracy by training 
classifiers on subsets of data from regions that have similar distributions of defects.
We propose adding a preprocessing stage to our platform that partitions the data by 
subdivision, computes the distribution of defects for each subdivision, and then groups
subdivisions based on a similarity metric for distributions. 

Currently we use a simple similarity metric--the Chi-squared distance--and group subdivision
based on a fixed similarity threshold. We plan to refine our similarity metric, use it
to explore the efficacy of clustering techniques for grouping subdivisions, and evaluate the
usefulness of the grouping by training and testing defect type classifiers on each such group. 

\section{Conclusions}
\section{Acknowledgments}

%
% The following two commands are all you need in the
% initial runs of your .tex file to
% produce the bibliography for the citations in your paper.
\bibliographystyle{abbrv}
\bibliography{bib}  % sigproc.bib is the name of the Bibliography in this case
% You must have a proper ".bib" file
%  and remember to run:
% latex bibtex latex latex
% to resolve all references
%
% ACM needs 'a single self-contained file'!
%
%APPENDICES are optional
%\balancecolumns
\end{document}
